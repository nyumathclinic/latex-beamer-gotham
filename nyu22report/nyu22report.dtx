% \iffalse
%<*driver>
\ProvidesFile{nyu22report.dtx}[2023/05/30 v0.2 NYU Official Report Template]
%</driver>
%<class>\ProvidesClass{nyu22report}[2023/05/30 v0.2 NYU Official Report Template]
%<*driver>
\documentclass{ltxdoc}
\usepackage{graphicx}
\usepackage{changepage}^^A      For the `adjustwidth' environment
\usepackage{hologo}^^A          Provides lots of logos


\usepackage{xcolor-nyu22}[2022/08/05]


% In the contemporary/subtle tone quadrant, we use sans for the main text and
% serif for the section titles.
%
% These font assignments have nothing to do with this package; they are part of
% the styling of a document
\renewcommand{\familydefault}{\sfdefault}
\usepackage{titlesec}
\usepackage{titling}
\newcommand{\headingfont}{\sffamily\color{NyuViolet}}
\titleformat*{\section}{\LARGE\headingfont}
\titleformat*{\subsection}{\Large\headingfont}
\titleformat*{\subsubsection}{\large\headingfont}
\renewcommand{\maketitlehooka}{\headingfont}

\usepackage{parskip}

\newcommand{\file}[1]{\texttt{#1}}

\EnableCrossrefs
\RecordChanges
\CodelineIndex
\usepackage{hyperref}
\begin{document}
  \DocInput{nyu22report.dtx}
\end{document}
%</driver>
% \fi

% \GetFileInfo{nyu22report.dtx} 
% \title{A \LaTeX{} class for the NYU Official Report Template}
% \author{Matthew Leingang\thanks{leingang@nyu.edu}} \date{\fileversion, Released \filedate}
% \maketitle

% \begin{abstract}
%  We describe and implement a \LaTeX{} class \texttt{nyu22report},
%  which is designed to match the report template provided by the NYU 
%  brand.
% \end{abstract}

% \changes{0.0}{2023/05/03}{First version}

% \section{Introduction}

% \section{Installation}
%
% \subsection{The class}
%
% To install the package, download the repository and within the module directory,
% execute the command: \texttt{l3build install}.
%
% \subsection{Fonts}
%
% Use the \pkg{nyu22fonts} package to select branded fonts.
%
% \section{Usage}
%
% \subsection{\hologo{XeLaTeX} and \hologo{LuaLaTeX} usage}
%
% \begin{verbatim}
%    \documentclass[options]{nyu22report}
% \end{verbatim}
% 
% \bibliographystyle{plain}
% \bibliography{nyu22report}
%
%
%
% \StopEventually{\PrintChanges}
%
% \section{Implementation}
%
% The class derives from the standard \pkg{report} class.
%
%    \begin{macrocode}
%<*class>
\DeclareOption*{% 
	\PassOptionsToClass{\CurrentOption}{report}%
}
\ProcessOptions\relax
\LoadClass[11pt]{report}
\RequirePackage{xcolor-nyu22}
%    \end{macrocode}
%
% \subsection{Metadata}
%
% \DescribeMacro{\subtitle}
% The \cmd{\title}, \cmd{\author}, and \cmd{\date} commands work the same way
% as in the standard classes. We additionally provide \cmd{\subtitle}:
%
% \begin{macro}{\subtitle}
% \changes{v0.1}{2023/05/30}{Added this command}
% Declare a subtitle for the document. It will be saved internally and
% typeset during \cmd{\maketitle}.
%    \begin{macrocode}
\DeclareRobustCommand*{\subtitle}[1]{\gdef\@subtitle{#1}}
%    \end{macrocode}
% \end{macro}
%
%
% \subsection{Fonts}
%
% We defer most of the font selection to a separate \pkg{nyu22fonts} package.
% These will set up the best selection of font families. This template
% uses sans serif font throughout.
%    \begin{macrocode}
\renewcommand{\familydefault}{\sfdefault}
%    \end{macrocode}
%
%    \begin{macrocode}
%</class>
%    \end{macrocode}
%
% \Finale
%
