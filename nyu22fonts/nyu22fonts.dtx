%
% \iffalse
%<*driver>
\ProvidesFile{nyu22fonts.dtx}
%</driver>
%<pkg>\ProvidesPackage{nyu22fonts}
%<*pkg>
  [2022/08/05 v0.9.1 NYU Typefaces and Fonts]
%</pkg>
%<*driver>
\documentclass{ltxdoc}
\usepackage{graphicx}
\usepackage{xcolor-material}^^A For the \colorsample command
\usepackage{changepage}^^A      For the `adjustwidth' environment

\usepackage{listings}
\lstset{language=bash,basicstyle=\ttfamily}

\usepackage{titling}
\pretitle{\begin{adjustwidth}{0pt}{-1in}\begin{flushleft}\Huge}
\posttitle{\par\end{flushleft}\end{adjustwidth}\vskip 0.5em}
\predate{\begin{flushleft}}
\postdate{\par\end{flushleft}}
\preauthor{\begin{flushleft}}
\postauthor{\par\end{flushleft}}

\usepackage{xcolor-nyu22}[2022/08/05]
\usepackage{nyu22fonts}


% In the contemporary/subtle tone quadrant, we use sans for the main text
% and serif for the section titles
% These font assignments have nothing to do with this package; they are part of
% the styling of a document
\renewcommand{\familydefault}{\sfdefault}
\usepackage{titlesec}
\newcommand{\headingfont}{\rmfamily\color{NyuViolet}}
\titleformat*{\section}{\LARGE\headingfont}
\titleformat*{\subsection}{\Large\headingfont}
\titleformat*{\subsubsection}{\large\headingfont}
\renewcommand{\maketitlehooka}{\headingfont}

\usepackage{parskip}


\EnableCrossrefs
\RecordChanges
\CodelineIndex
\usepackage{hyperref}
\begin{document}
  \DocInput{nyu22fonts.dtx}
\end{document}
%</driver>
% \fi

% \GetFileInfo{nyu22fonts.dtx} 
% \title{NYU Typefaces and Fonts}
% \author{Matthew Leingang\thanks{leingang@nyu.edu}} \date{\fileversion, Released \filedate}
% \maketitle

% \begin{abstract}
%  This package will load fonts for the NYU brand.
%  Package options will allow choices among the primary (proprietary) faces,
%  open (Google) faces, system (Microsoft) faces, and T1 fonts for those who
%  must use \texttt{pdftex}. 
% \end{abstract}

% \changes{unreleased}{2019/12/10}{First working release}
%

% \section{Our Typographic Style}
%
% NYU’s typographic language brings your communication to life. Like color, the
% fonts we use reinforce the tone of our communications and designs.
%
% Gotham and Mercury Text are NYU’s two typefaces. This versatile group of font
% families can be combined to achieve different tones. That flexibility helps
% our communications appeal to many of our different audiences, including
% students, parents, alumni, faculty and staff, peers, and supporters, while
% maintaining a thematically consistent brand. Font choice also establishes a
% clear hierarchy of information, allowing audiences to easily navigate your
% communications.
%
% \textbf{Gotham} references the no-nonsense signage of New York City. It’s a
% typeface that’s meant to feel familiar and approachable but strong enough to
% grab and hold your attention within the busy city.
%
% \textbf{Mercury Text} is a high performance serif typeface born from nearly a decade of
% research and development. Mercury Text is resilient enough to work in a wide
% variety of communications.
%
%
% \StopEventually{\PrintChanges}
%
% \section{Implementation}
%
%    \begin{macrocode}
%<*pkg>
%    \end{macrocode}
%
%    \begin{macrocode}
% Set up the NYU primary font families
\RequirePackage{fontspec}
\setmainfont{Mercury Text G2 Roman}[
  BoldFont=Mercury Text G2 Bold,
  ItalicFont=Mercury Text G2 Italic
]
\setsansfont{Gotham Book}[
  BoldFont=Gotham Medium,
  ItalicFont=Gotham Book Italic
]
%    \end{macrocode}
%
%    \begin{macrocode}
%</pkg>
%    \end{macrocode}
%
% \Finale
%
