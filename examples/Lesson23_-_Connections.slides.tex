%%!TEX TS-program = xelatexmk
\documentclass
[ignorenonframetext,14pt,aspectratio=169]
{ngelessonslides}
%%
\title[
lesson number=23,
textbook section={49}
]{Connection}
\course[code=MATH-UA 120,section=005]{Discrete Mathematics}
\term[code=2019F]{Fall 2019}
\author{Prof. Leingang}
\usepackage[us]{datetime}
\newdate{lessondate}{5}{12}{2019}
\edef\lessondateiso{\getdateyear{lessondate}-\twodigit{\getdatemonth{lessondate}}-\twodigit{\getdateday{lessondate}}}
\date{\displaydate{lessondate}}
\makeatletter
\keyword{course: \ngelesson@course@code\space\insertcoursename}
\keyword{term: \ngelesson@term}
\keyword{MT\ngelesson@textbooksection}% must fix if chap > 9
\keyword{\insertcoursename}
\makeatother
\keywords{}
\lessondescription{}% Oliver's description
\usepackage{siunitx}
\usepackage{overpic}
\usepackage{snapshot}
\usepackage{xr}
\usepackage{zref-xr}
\usepackage{textcomp}
\usetheme{Gotham}

\let\im\relax
\usepackage{discrete}
\usetikzlibrary{patterns}
\colorlet{primary}{cyan}
\tikzset{onslide/.code args={<#1>#2}{%
  \only<#1>{\pgfkeysalso{#2}}
}}

\tikzset{simple graph/.style={
    every edge/.append style={thick},
    dot/.style={draw,circle,inner sep=0pt,minimum width=1ex},
    vertex/.style={dot},
}}
\newcommand{\tikzmark}[1]{\tikz[overlay,remember picture] \node (#1) {};}
\usetikzlibrary{decorations}

\begingroup
\usebeamercolor{block title example}
\xglobal\colorlet{example}{bg}
\endgroup


\usepackage{hyperref}
\makeatletter
\hypersetup{
pdftitle=\ngelesson@longtitle,
pdfauthor=\insertauthor,
pdfsubject=\insertcoursename,
pdfkeywords=\the\ngelesson@keywords,
}
\makeatother
\begin{document}

\maketitle

\begin{frame}%
    \maketitle
    \begin{tikzpicture}[remember picture,overlay]
        \node[
anchor=south west, at={([xshift=1mm,yshift=1mm]current page.south west)}
        ]
        {\includegraphics[width=2.5in]{NYUCourant}};
    \end{tikzpicture}
\end{frame}

\section*{Announcements}
\timecheck{\LARGE\clock{11}{00}}

\begin{frame}{It's the final countdown}%
\begin{itemize}
\item Dec 3: Quiz on §36 and §47; Cover §48 Subgraphs
\item Dec 5: §49 Connection
\item Dec 6: HW on §36 and §47 due
\item Dec 10: Quiz on §§48–49; Cover §50 Trees
\item Dec 12: §51 Eulerian Graphs
\item Dec 13: HW on §§48–49 not due;
\item Dec 17, 8:00 AM in CIWW 101: Review for final
\item Dec 19, 8:00 AM: Final Exam
\end{itemize}
\end{frame}

\begin{frame}{Motivation}
\begin{itemize}
\item We like road systems that allow us to get from point $A$ to point $B$ for all $A$ and $B$.
\item The importance of the internet is that any computer can send a message to any other computer
\item Kevin Bacon numbers for actors, Erd\H os numbers for mathematicians
\item An electrical company may need to identify its most critical lines
\end{itemize}
\end{frame}

\mode<article>{%
\section*{Objectives}
}% end mode<article>

\begin{frame}{Objectives}
\begin{columns}
\begin{column}{0.65\textwidth}
\begin{itemize}
    \item Define and use the is-connected-to relation on the set of vertices of a graph
    \item Define and identify the connected components of a graph
    \item Define and identify the cut edges of a graph.
\end{itemize}
\end{column}
\begin{column}{0.25\textwidth}
\mode<presentation>{\includegraphics[width=\textwidth]{Anonymous_target_with_arrow}}
\end{column}
\end{columns}
\end{frame}

\subsection*{Notes}

See also:
\begin{itemize}%%%
\item Fall 2019: tweaking colors
\item Spring 2019: slidified.  Still pretty thin; needs worksheet and pictures.
\item Spring 2018: \href{https://www.evernote.com/shard/s2/nl/210343/c9a0520f-8e6d-42fa-b0e2-8056a0964ece/}{Notes in Evernote}
\end{itemize}

\section*{Materials}

\begin{itemize}
\item Colored chalk/markers
\end{itemize}

\mode<article>{
\section*{Do Now}}

\begin{frame}[label=do-now]
\begin{columns}
\begin{column}{0.45\textwidth}
\begin{exampleblock}<+->{Do Now}%%
Let $G$ be the graph below.
\begin{center}
     \begin{tikzpicture}[scale=0.4,simple graph]
     \begin{scope}[x={(0cm,1bp)},y={(-1bp,0cm)}]
     \begin{scope}[every node/.append style={vertex}]
         \draw (173.0,306.0) node[label=$1$] (n0) {};
         \draw (27.0,234.0)  node[label=180:$2$] (n1) {};
         \draw (99.0,234.0)  node[label=$3$] (n2) {};
         \draw (82.0,90.0)   node[label=$4$] (n3) {};
         \draw (137.0,162.0) node[label=-25:$5$] (n4) {};
         \draw (154.0,90.0)  node[label=$6$] (n5) {};
         \draw (118.0,18.0)  node[label=45:$7$] (n6) {};
         \path (n0) edge (n1) edge (n4) edge (n5) edge[out=15,in=105] (n6)
               (n1) edge (n4) edge[out=-15,in=255] (n6)
               (n2) edge (n3) edge (n4)
               (n3) edge (n6)
               (n4) edge (n5) edge (n6)
               (n5) edge (n6)
          ;
     \end{scope}
     \end{scope}
     \end{tikzpicture}
\end{center}
Find $\omega(G)$ and $\alpha(G)$.
\end{exampleblock}
\end{column}
\begin{column}{0.45\textwidth}
\begin{solution}<+->
\begin{itemize}
\item<.->     There is a clique of size $4$: $\set{1,5,2,7}$.
    But there are no cliques of size $5$.  That would require five vertices of degree at least $4$,
    and there are only two ($1$ and $7$).
    So $\omega(G) = 4$.
\item<+->
    There are two independent sets of size $3$:
        $\set{2,3,6}$ and $\set{2,4,6}$.
    So $\alpha(G) = 3$.
\end{itemize}
\end{solution}
\end{column}
\end{columns}
\end{frame}

\mode<article>{%
\section*{Anticipatory Set}}
\timecheck{\LARGE\clock{09}{15}}

\tableofcontents

\mode<article>{%
\section*{Procedure}}
\timecheck{\LARGE\clock{09}{00}}

\section{Walks}

\begin{frame}[label=walk-def]
\begin{definition}
    Let $G=(V,E)$ be a graph.
    A \defined{walk} in $G$ is a list (or finite sequence) of vertices
    with each adjacent to the next.  We write it as
    \[
        W = (v_0,v_1,\dots,v_\ell)
    \]
    where $v_i \in V$ for all $i$, and $v_{i-1} \sim v_i$ for each $i$ from $1$ to $\ell$.
\end{definition}
\begin{itemize}
\item We sometimes write a walk with the vertices separated by $\sim$:
\[
    W = v_0 \sim v_1 \sim \dots \sim v_\ell
\]
\item The \defined{length} of a walk is the number of edges traversed in the sequence
($\ell$)
\end{itemize}
\end{frame}

\begin{frame}[label=walk-ex]%
\begin{example}<+->
{\usebeamercolor{block title example}\xglobal\colorlet{efg}{bg}}
Using the same graph as in the Do Now, here are some walks:
\begin{columns}
\begin{column}{0.45\textwidth}
    \centering%
     \begin{tikzpicture}[scale=0.4,simple graph,
         walk/.style={preaction={draw,ultra thick,efg}}
     ]
     \begin{scope}[x={(0cm,1bp)},y={(-1bp,0cm)}]
     \begin{scope}[every node/.append style={vertex}]
         \draw (173.0,306.0)
               node[onslide={<2,3,4,5>{walk}},label=$1$] (n0) {};
         \draw (27.0,234.0)
               node[onslide={<4,5>{walk}},label=180:$2$] (n1) {};
         \draw (99.0,234.0)
               node[label=$3$] (n2) {};
         \draw (82.0,90.0)   node[label=$4$] (n3) {};
         \draw (137.0,162.0)
               node[onslide={<3,4,5>{walk}},label=-25:$5$] (n4) {};
         \draw (154.0,90.0)
               node[onslide={<3,5>{walk}},label=$6$] (n5) {};
         \draw (118.0,18.0)
               node[onslide={<3,5>{walk}},label=45:$7$] (n6) {};
         \path (n0) edge[onslide={<4>{walk}}] (n1)
                    edge[onslide={<3,4,5>{walk}}] (n4) edge (n5)
                    edge[out=15,in=105,onslide={<0>{walk}}] (n6)
               (n1) edge[onslide={<4,5>{walk}}] (n4) edge[out=-15,in=255] (n6)
               (n2) edge (n3) edge (n4)
               (n3) edge (n6)
               (n4) edge[onslide={<3,5>{walk}}] (n5)
                    edge[onslide={<5>{walk}}] (n6)
               (n5) edge[onslide={<3,5>{walk}}] (n6)
          ;
     \end{scope}
     \end{scope}
     \end{tikzpicture}
\end{column}
\begin{column}{0.45\textwidth}
\begin{itemize}
\item<+-> $1$
\item<+-> $1 \sim 5 \sim 6 \sim 7$
\item<+-> $1 \sim 5 \sim 2 \sim 1$
\item<+-> $1 \sim 5 \sim 6 \sim 7 \sim 5 \sim 2$
\end{itemize}
\end{column}
\end{columns}

\end{example}
\end{frame}

\Do{Remarks:
\begin{itemize}
\item Walks can go nowhere.
\item Walks can visit a vertex more than once.
\item Walks can be reversed.
\item Walks can come back to where they started.
\item Walks must traverse edges.
\end{itemize}
}

\begin{frame}[label=concat-def]
\begin{definition}
Suppose that $W_1 = v_0 \sim v_1 \sim \dots \sim v_\ell$ and
$W_2 = w_0 \sim w_1 \sim \dots \sim w_k$
are walks with $v_\ell = w_0$.  The \defined{concatenation} of $W_1$ and $W_2$
is the walk
\[
   W_1 + W_2 = v_0 \sim v_1 \sim \dots \sim (v_\ell = w_0) \sim w_1 \sim \dots \sim w_k
\]
\end{definition}
\begin{itemize}
\item This walk has length $\ell+k$.
\item Not every pair of walks can be concatenated
\item Even if you can concatenate in the opposite order, walk concatenation is not usually commutative
\end{itemize}
\end{frame}

\section{Paths}

\begin{frame}[label=path-def]
\begin{definition}
    If $u$ and $v$ are vertices, a \defined{walk from $u$ to $v$} or \defined{$(u,v)$-walk} is a walk that starts at $u$ and ends at $v$:
    \[
        W = u \sim \dots \sim v
    \]
\end{definition}
\begin{definition}
    A \defined{path} in $G$ is a walk in which no vertex is repeated.
    A \defined{path from $u$ to $v$} or \defined{$(u,v)$-path} is a path that starts at $u$ and ends at $v$.
\end{definition}
\end{frame}

\begin{frame}[label=path-prop]
\begin{block}<+->{Proposition 49.4}
    Let $P$ a path in a graph $G$.  then $P$ does not traverse any edge of $G$ more than once.
\end{block}
\begin{proof}<+->
Suppose $P$ traverses an edge $uv$ twice.  The possibilities are (without loss of generality):
\begin{gather*}
 \dots \sim u \sim v \sim \dots \sim u \sim v \sim \dots \\
\dots \sim u \sim v \sim \dots \sim v \sim u \sim \dots \\
\dots \sim u \sim v \sim u \sim \dots
\end{gather*}
Each of these repeats a vertex ($u$ or $v$), so is not a path.
\end{proof}
\end{frame}

\begin{frame}[label=path-graph-def]
We defined a path to be a sequence of vertices, but the edges joining the vertices
can also be included to form a graph.  So when convenient, we'll consider paths as graphs too.
\begin{definition}
    Let $n$ be a positive integer.
    The \defined{path graph} on $n$ vertices is the graph $P_n$ whose vertices
    are $\set{v_1,v_2,\dots,v_n}$, and whose edges are
    $\set{\set{v_1,v_2},\set{v_2,v_3},\dots,\set{v_{n-1},v_n}}$.
\end{definition}
\end{frame}

\begin{frame}[label=ctdto-def]
\begin{definition}[Connected to]
    Let $G=(V,E)$ be a graph.
    We say that $u$ is \defined{connected to} $v$ if there exists a $(u,v)$-path in $G$.
\end{definition}
\end{frame}

\begin{frame}[label=ctdto-ex]
    \begin{columns}
    \begin{column}{0.45\textwidth}
        \begin{example}<+->
        Find the pairs of connected vertices in the graph below.
        \begin{center}
            \begin{tikzpicture}[simple graph]
                \begin{scope}[every node/.append style={vertex}]
                    \path (0,0) node[label=180:$1$] (v1) {}
                        ++(1,0) node[label=0:$2$] (v2) {}
                        ++(0,1) node[label=0:$3$] (v3) {}
                        ++(-1,0) node[label=180:$4$] (v4) {};
                    \path (v1) edge (v2) edge (v4)
                          (v3) edge (v2) edge (v4);
                    \begin{scope}[xshift=2cm]
                    \path (0,0) node[label=$9$] (v9) {}
                        ++(1,1) node[label=$0$] (v0) {};
                    \path (v9) edge (v0);
                    \end{scope}
                    \begin{scope}[yshift=2cm]
                    \path (0,0) node[label=$5$] (v5) {};
                    \end{scope}
                    \begin{scope}[xshift=2cm,yshift=2cm]
                    \path (0,0) node[label=$6$] (v6) {}
                          (1,0) node[label=$7$] (v7) {}
                          (60:1) node[label=$8$] (v8) {};
                    \draw (v6) -- (v7) -- (v8);
                    \end{scope}
                \end{scope}
            \end{tikzpicture}
        \end{center}
        \end{example}
    \end{column}
    \begin{column}{0.45\textwidth}
        \begin{solution}<+->
            \centering
            \begin{tabular}{r|l}
                vertex & connected to \\\hline
                0 & 0, 9 \\
                1 & 1, 2, 3, 4 \\
                2 & 2, 3, 4, 1 \\
                3 & 3, 4, 1, 2 \\
                4 & 4, 1, 2, 3 \\
                5 & 5 \\
                6 & 6, 7, 8 \\
                7 & 7, 8, 6 \\
                8 & 8, 7, 6 \\
                9 & 9, 0
            \end{tabular}
        \end{solution}
    \end{column}
    \end{columns}
\end{frame}

\begin{frame}[label=ctd-adj]%%
    \begin{alertblock}{Note}
       It's tempting to use “connected” as a synonym for \emph{adjacent} when describing graphs.
       But they are different terms.
       \begin{itemize}
           \item If $v$ is adjacent to $w$, then $v$ is connected to $w$.
           \item But not vice versa!
       \end{itemize}
    \end{alertblock}
\end{frame}

\begin{frame}[label=ctd-thm]
\begin{block}{Theorem 49.8}
    Let $G$ be a graph.  The is-connected-to relation is an equivalence relation on the set of vertices of $G$.
\end{block}
\end{frame}

\begin{frame}[label=ctd-thm-proof-1]
To prove this, we need to show the relation is: reflexive, symmetric, and transitive.
\begin{itemize}
\item<+-> \structure{Reflexive}
\uncover<+->{Let $v \in V$.  The walk $(v)$ is a path, because it does not repeat any vertices.
    This path starts and ends at $v$, so $v$ is connected to $v$.}
\item<+-> \structure{Symmetric}
\uncover<+->{Let $u$ and $v$ be connected vertices, and $P$ a path from $u$ to $v$.
    Denote by $P^{-1}$ the walk that lists the same vertices in the opposite order.
    This is a path, in fact a $(v,u)$-path.  So $v$ is connected to $u$.}
\item<+-> \structure{Transitive}
\uncover<+->{Suppose that $u$ is connected to $v$, and $v$ is connected to $w$.
    Let $W_1$ be a $(u,v)$-path, and $W_2$ a $(v,w)$-path.}
\uncover<+->{Then $W_1+W_2$ is a $(u,w)$-\alert{walk}. It could repeat vertices!}
\end{itemize}
\end{frame}

\begin{frame}[label=lemma]
\begin{block}<+->{Lemma 49.7}
Let $G$ be a graph and $x$ and $y$ vertices.
If there is an $(x,y)$-walk in $G$, then there is an $(x,y)$-path in $G$.
\end{block}
\begin{proof}<+->
\begin{itemize}
\item<.-> Consider the set of all $(x,y)$-walks in $G$.
\item<+-> It's nonempty, so there exists one with minimal length (WOP).  Call it $P$; we claim it's a path.
\item<+-> If $P$ does repeat a vertex, we can eliminate the portion between the two instances, creating a shorter path. \raa \qedhere
\end{itemize}
\end{proof}
\end{frame}

\begin{frame}[label=lemma-2]
More about shortening that path:  Suppose $P$ is a walk from $x$ to $y$ that repeats $u$.  Then
\[
    P = \underbrace{x \sim \dots \sim u}_{W_1}%
    \underbrace{\sim \dots \sim u}_{W_2}%
    \underbrace{\sim \dots \sim y}_{W_3}
\]
We are assuming that $W_2$ has positive length.
Let $P' = W_1 + W_3$.  Then $P'$ is also an $(x,y)$-walk, but has a shorter length than $P$.
\end{frame}

\begin{frame}[label=ctd-thm-proof-2]
Now to show is-connected-to is transitive.
\begin{itemize}
\item Suppose that $u$ is connected to $v$, and $v$ is connected to $w$.
    Let $W_1$ be a $(u,v)$-path, and $W_2$ a $(v,w)$-path.
    Then $W_1+W_2$ is a $(u,w)$-walk.
    By the lemma, there is a $(u,w)$-path in $G$.
    So $u$ is connected to $v$.
\item Therefore is-connected-to is an equivalence relation.
\end{itemize}
\end{frame}

\begin{frame}[label=cpt-def]
\begin{definition}
    A \defined{component} of $G$ is a subgraph of $G$ induced by an equivalence class of
    the is-connected-to relation on $V(G)$.
\end{definition}
In other words, a component is:
\begin{itemize}
\item A maximal set of vertices, each of which is connected to the rest;
\item plus all the edges that join them.
\end{itemize}
\end{frame}

\begin{frame}[label=cpt-ex]
    \begin{columns}
    \begin{column}{0.45\textwidth}
        \begin{example}<+->
        Find the components in the graph below.
        \begin{center}
            \begin{tikzpicture}[simple graph,
                     cpt/.style={preaction={draw,ultra thick,example}}
            ]
                \begin{scope}[
                    vertex/.append style={onslide={<4,9>{cpt}}},
                    every edge/.append style={onslide=<9>{cpt}}
                ]
                    \path (0,0) node[vertex,label=180:$1$] (v1) {}
                        ++(1,0) node[vertex,label=0:$2$] (v2) {}
                        ++(0,1) node[vertex,label=0:$3$] (v3) {}
                        ++(-1,0) node[vertex,label=180:$4$] (v4) {};
                    \path (v1) edge (v2) edge (v4)
                          (v3) edge (v2) edge (v4);
                \end{scope}
                \begin{scope}[xshift=2cm,
                    vertex/.append style={onslide={<7,12>{cpt}}},
                    every edge/.append style={onslide={<12>{cpt}}}
                ]
                \path (0,0) node[vertex,label=$9$] (v9) {}
                    ++(1,1) node[vertex,label=$0$] (v0) {};
                \path (v9) edge (v0);
                \end{scope}
                \begin{scope}[yshift=2cm,
                    vertex/.append style={onslide={<5,10>{cpt}}}
                ]
                \path (0,0) node[vertex,label=$5$] (v5) {};
                \end{scope}
                \begin{scope}[xshift=2cm,yshift=2cm,
                    vertex/.append style={onslide={<6,11>{cpt}}},
                    every edge/.append style={onslide=<11>{cpt}}
                ]
                \path (0,0) node[vertex,label=$6$] (v6) {}
                      (1,0) node[vertex,label=$7$] (v7) {}
                      (60:1) node[vertex,label=$8$] (v8) {};
                \draw (v6) edge (v7) edge (v8) (v7) edge (v8);
                \end{scope}
            \end{tikzpicture}
        \end{center}
        \end{example}
    \end{column}
    \begin{column}{0.45\textwidth}
        \begin{solution}<+->
            \begin{itemize}
            \item<+-> Equivalence classes:
                  \uncover<+->{\textcolor<.>{example}{$\set{1,2,3,4}$},}
                  \uncover<+->{\textcolor<.>{example}{$\set{5}$},}
                  \uncover<+->{\textcolor<.>{example}{$\set{6,7,8}$},}
                  \uncover<+->{\textcolor<.>{example}{$\set{9,0}$}.}
            \item<+-> Each of these induces a component (subgraph)
            \end{itemize}
        \end{solution}
    \end{column}
    \end{columns}
\end{frame}

\begin{frame}[label=ctd-def]
\begin{definition}[Connected]
    A graph is called \defined{connected} if it has exactly one component.
\end{definition}
\begin{itemize}
    \item In other words, $G$ is connected iff for all vertices $x,y\in V$,
    there exists an $(x,y)$-path in $G$.
    \item $G$ is \emph{not} connected iff there exist $x,y\in V$ such that
    there \emph{does not} exist an $(x,y)$-path in $G$.
\end{itemize}
\end{frame}

\section{Disconnection}

\begin{frame}[label=cut-def]
    \begin{definition}[Cut vertex, cut edge]
        Let $G$ be a graph.
        \begin{itemize}
            \item A vertex $v$ is called a \defined{cut vertex} of $G$ if $G-v$ has more components than $G$.
            \item An edge $e$ is called a \defined{cut edge} of $G$ if $G-e$ has more components then $G$.
        \end{itemize}
    \end{definition}
    \begin{itemize}
        \item If $e=xy$, then $e$ is a cut edge of $G$ iff
              there is no $(x,y)$-path in $G$ other than $x \sim y$.
        \item $v$ is a cut vertex iff there exist $x$ and $y$ adjacent to $v$
              such that all $(x,y)$-paths contain $v$.
    \end{itemize}
\end{frame}

\begin{frame}[label=cut-ex]
    \begin{columns}
    \begin{column}{0.45\textwidth}
        \begin{example}<+->
            Find all the cut vertices and edges of the graph
            below.
            \begin{center}
                \begin{tikzpicture}[simple graph]
                % two triangles joined by an edge
                    \path node[vertex,label=$c$] (c) {}
                    (60:1) node[vertex,label=$a$] (a) {}
                    (120:1) node[vertex,label=$b$] (b) {}
                    (0,-1) node[vertex,label=-90:$d$] (d) {}
                    ++(-60:1) node[vertex,label=0:$f$] (f) {}
                    ++(-1,0) node[vertex,label=180:$e$] (e) {}
                    (a) edge (b) edge (c)
                    (b) edge (c)
                    (c) edge (d)
                    (d) edge (e)
                    (e) edge (f)
                    (f) edge (d);
                \end{tikzpicture}
            \end{center}
        \end{example}
    \end{column}
    \begin{column}{0.45\textwidth}
        \begin{solution}<+->
            \begin{itemize}
                \item<.-> $c$ and $d$ are cut vertices.
                \item<+-> $cd$ is the only cut edge.
            \end{itemize}
        \end{solution}
    \end{column}
    \end{columns}
\end{frame}

\begin{frame}[label=cut-prac]
\begin{exampleblock}<+->{Practice}
Find the cut vertices and edges of the graphs below.
\begin{columns}
\begin{column}{0.45\textwidth}
    \begin{center}
        \begin{tikzpicture}[simple graph]
            \path node[vertex,label=$a$] (a) {}
            ++( 1,-1) node[vertex,label=$c$] (c) {} edge (a)
            ++( 1,-1) node[vertex,label=$f$] (f) {} edge (c)
            ++( 1, 1) node[vertex,label=$e$] (e) {} edge (f)
            ++(-1, 1) node[vertex,label=$d$] (d) {} edge (e) edge (c)
            ++(-2,-2) node[vertex,label=135:$b$] (b) {} edge (c) edge (a);
            \node[at=(current bounding box.south east),anchor=south east] {$G$};
        \end{tikzpicture}
    \end{center}
\end{column}
\begin{column}{0.45\textwidth}
    \begin{center}
        \begin{tikzpicture}[simple graph]
            \path node[vertex,label=$a$] (a) {}
            ++( 1, 0) node[vertex,label=$b$] (b) {} edge (a)
            ++( 0,-1) node[vertex,label=left:$c$] (c) {} edge (b)
             +( 0,-1) node[vertex,label=left:$d$] (d) {} edge (c)
            ++( 1, 0) node[vertex,label=135:$e$] (e) {} edge (c)
             +( 0, 1) node[vertex,label=$f$] (f) {} edge (e)
             +( 1, 0) node[vertex,label=0:$g$] (g) {} edge (e) edge (f)
            ++( 0,-1) node[vertex,label=-90:$i$] (i) {} edge (e)
            ++( 1, 0) node[vertex,label=-90:$h$] (h) {} edge (i)
            ;
            \node[at=(current bounding box.south west),anchor=south west] {$H$};
        \end{tikzpicture}
    \end{center}
\end{column}
\end{columns}
\end{exampleblock}
\begin{solution}<+->
\begin{tabular}{c|c|c}
graph & cut vertices & cut edges \\\hline
$G$   & $c$          & $\emptyset$ \\
$H$   & $c$, $e$, $i$ & $ab$, $bc$, $cd$, $ce$, $ei$, $hi$
\end{tabular}
\end{solution}
\end{frame}

\begin{example}
Find the cut edges and vertices on the graphs below:
See \href{https://math.stackexchange.com/q/457159/2785}{MSE}
\begin{center}
\includegraphics[width=\textwidth]{dRD1r}
\end{center}
\end{example}

\begin{frame}[label=cut-thm]
    \begin{block}{Theorem 49.12}
   Let $G$ be a connected graph and suppose $e$ is a cut edge of $G$.
   Then $G-e$ has \emph{exactly} two components.
   \end{block}
   Note $G-e$ has \emph{at least} two components by definition of cut edge.
   We need to show it can't have more.
\end{frame}

\begin{frame}[label=cut-proof]
\begin{proof}
\begin{overlayarea}{\textwidth}{0.8\textheight}
\begin{itemize}
\item<only@1-4|uncover@+-> Suppose FTSOC that $G-e$ has at least three components.  Then there exist vertices $a$, $b$, and $c$ in $G$,
no pair of which are connected in $G-e$.
\item<only@1-4|uncover@+-> There is a path $P$ from $a$ to $b$ in $G$, though.  Such a path \textbf{must traverse the edge $e$}.
\item<only@1-4|uncover@+-> There is a path $Q$ from $c$ to $a$ in $G$, too.  It must also traverse $e$.
\item<uncover@+-> Let $e$ join vertices $x$ and $y$ in the direction of $P$.  Does $Q$ visit $x$ or $y$ first?
\begin{itemize}
\item<uncover@+-> If $Q$ visits $x$ first, we can walk along $Q$ from $c$ to $x$, then along $P^{-1}$ from $x$ to $a$.
    This is a walk from $c$ to $a$ skipping $e$, hence a $(c,a)$-walk in $G-e$.  Therefore $c$ is connected to $a$ in $G-e$.
    \raa
\item<uncover@+-> If $Q$ visits $y$ first, we can walk along $Q$ from $c$ to $y$, then along $P$ from $c$ to $b$.
    This is a walk from $c$ to $b$ skipping $e$, hence a $(c,b)$-walk in $G-e$.  Therefore $c$ is connected to $b$ in $G-e$.
    \raa
\end{itemize}
\item<uncover@+-> Either way, we have a contradiction, so $G-e$ cannot have three components.\qedhere
\end{itemize}

\end{overlayarea}\end{proof}
\end{frame}

\mode<article>{%
\section*{Guided Practice}}

See practice elements above.

\mode<article>{%
\section*{Closure}}
\timecheck{\LARGE\clock{10}{40}}

\begin{frame}[label=summary]{Summary of §49}
\begin{itemize}
\item walk and path.  A path is a walk that doesn't repeat vertices.
\item connected (equivalence relation on vertices of a graph), components (subgraphs induced by equivalence classes).  Each of these is easier to “see” than to describe.
\item Cut vertices and cut edges separate the graph when removed.  Cut vertices can separate the graph into many components, but cut edges only two.
\end{itemize}
\end{frame}

\mode<article>{%
\section*{Independent Practice}

\subsection*{Practice Problems}

All problems from Scheinerman.

\begin{itemize}
    \item §49:
\end{itemize}

\subsection*{Homework problems}

}% end mode<article>

\bibliographystyle{amsalpha}
\bibliography{undergrad-textbooks}

\end{document}
\endinput
%%
%% End of file `Lesson23_-_Connections.slides.tex'.
